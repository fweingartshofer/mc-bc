%! Author = florian
%! Date = 25.05.23

\section{Introduction}\label{sec:introduction}
With the boom of blockchains, more and more people are mining blocks for different blockchains.
Bitcoin alone is estimated to consume around 143 terawatt-hours per year according to the Cambridge Bitcoin Electricity Consumption Index\footnote{\url{https://ccaf.io/cbnsi/cbeci}, last accessed on 2023-05-25}.

It requires large amounts of energy because of its consensus mechanism, proof of work~\cite{why-does-bitcoin-use-so-much-energy}.
This algorithm solves complex mathematical equations in order to mine a block, which requires significant computational resources, thereby consuming more energy.
To save energy, other consensus mechanisms exist, such as proof-of-space-and-time or proof-of-stake.

This paper is split into three sections:
\begin{itemize}
    \item \textbf{Eco-Friendly Consensus Mechanisms}: This section describes alternative consensus mechanisms to proof of work.
    \item \textbf{Eco-Friendly Blockchains}: This section lists and describes blockchains using alternative consensus mechanisms, which are discussed in Section~\ref{sec:eco-friendly-consensus-mechanisms}.
    \item \textbf{Conclusion}: This section will conclude if eco-friendly blockchains are a suitable alternative.
\end{itemize}
