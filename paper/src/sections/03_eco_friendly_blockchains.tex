%! Author = florian
%! Date = 29.05.23

\section{Eco-Friendly Blockchains}\label{sec:alternative-blockchains}

\subsection{Ethereum}\label{subsec:ethereum}

\subsection{The Chia Network}\label{subsec:chia}

\subsection{The Energy Web Chain}\label{subsec:the-energy-web-chain}
The Energy Web Chain is a blockchain based virtual machine with the energy sector as target audience.
The organisation developing the energy web chain is called Energy Web.
Per the Energy Web website: ``Energy Web is a global non-profit on a mission to accelerate the energy transition by developing and deploying open-source Web3 technologies that help companies unlock business value from clean and distributed energy resources.''\footnote{\url{https://www.energyweb.org/why-we-exist/} last accessed on 2023-06-13}.
What the energy transition actually consists of is not mentioned.
The Energy Web Chain is compatible with the Ethereum virtual machine and ERC20 tokens.
The blockchain uses proof of authority as consensus mechanism, which is described in subsection\ \ref{subsec:proof-of-authority}.
Every validator gets Energy Web Tokens for creating blocks as an incentive to stay honest.

\subsubsection{SolarCoin}
One token that is based on the Energy Web Chain is the SolarCoin.
It uses proof of generation as consensus mechanism.
For every MWh generated by a photovoltaic system the producer gets one SolarCoin.
It is essentially a cryptocurrency that is backed by solar energy.
% TODO: Citation