%! Author = florian
%! Date = 29.05.23

\section{Eco-Friendly Blockchains}\label{sec:alternative-blockchains}
This section will describe two alternative blockchains, which claim to be eco-friendly.
Ethereum should be mentioned as a blockchain that moved from a proof of work consensus mechanism to a proof of stake consensus mechanism, which is described in subsection\ \ref{subsec:proof-of-stake}.
Ethereum was able to reduce its power consumption by 99.99\% after migrating to a proof of stake based consensus mechanism according to the Cambridge Blockchain Sustainability Index.\cite{CBECI}

\subsection{The Chia Network}\label{subsec:chia}

\subsection{The Energy Web Chain}\label{subsec:the-energy-web-chain}
The Energy Web Chain is a blockchain based virtual machine with the energy sector as target audience.
The organisation developing the energy web chain is called Energy Web.
Per the Energy Web website: ``Energy Web is a global non-profit on a mission to accelerate the energy transition by developing and deploying open-source Web3 technologies that help companies unlock business value from clean and distributed energy resources.''\footnote{\url{https://www.energyweb.org/why-we-exist/} last accessed on 2023-06-13}.
What the energy transition actually consists of is not mentioned.
The Energy Web Chain is compatible with the Ethereum virtual machine and ERC20 tokens.
The blockchain uses proof of authority as consensus mechanism, which is described in subsection\ \ref{subsec:proof-of-authority}.
Every validator gets Energy Web Tokens for creating blocks as an incentive to stay honest.\cite{kraken-ewc}

There are multiple use cases that Energy Web wants the Energy Web Chain for.
Apart for hosting other tokens that are related to the energy sector, one of the being SolarCoin\ \ref{subsubsec:solarcoin}, they want to provide a number of different solutions.

One interesting solution is called Green Proofs.
Green Proofs try to solve the problem of trusting ESG reports and sustainability claims without verifying these claims.
ESG reporting is a type of corporate disclosure regarding environmental, social and governance(ESG) efforts and claims.
Often investors want an ESG report to make decisions.
It is often used for branding, since customers choose to do business with corporations that align with their values.
Furthermore, ESG reports are often required by regulators in order to see if an organisation is compliant with regulations.
Energy Web Green Proofs is a solution to track low-carbon products throughout supply chains.
Users are then able to verify ESG reports without having to trust them.\cite{ESG-reporting, EW-green-proofs}

\subsubsection{SolarCoin}\label{subsubsec:solarcoin}
One token that is based on the Energy Web Chain is the SolarCoin.
It uses proof of generation as consensus mechanism, which is a consensus mechanism based on proof of authority, since it requires a central authority to verify that something was actually generated.
In the case of SolarCoin the proof of generation is the number of generated MWh-
For every MWh generated by a photovoltaic system the producer gets one SolarCoin.
It is essentially a cryptocurrency that is backed by solar energy.\cite{solarcoin-save-the-planet}